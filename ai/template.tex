\documentclass{jlreq}

\usepackage{amsmath}
\usepackage{bm}
\usepackage{fancyhdr}
\usepackage{float}
\usepackage{graphicx}
\usepackage{physics}
\usepackage{siunitx}

\numberwithin{equation}{section}

\pagestyle{fancy}
\fancyhf{}
\fancyhead[R]{\thepage}

\begin{document}

\tableofcontents
\clearpage

% 1週目のみ
\section{実験全体の目的}
\section{実験器具}
使用した機材は,Dell\ Inspiron\ 15\ 3535である.以下に細かい仕様を記載する.

\begin{description}
  \item[OS] Windows11\_Home
  \item[CPU] AMD Ryzen 5 7530U with Radeon Graphics 2.00GHz
  \item[RAM] 16.0 GB
  \item[Eclipse] 
\end{description}

\section{実験方法}

\section{結果}

\section{考察}

\begin{thebibliography}{9}
  \item 野宮浩揮.プロジェクト実習Ⅲ 人工知能 実験テキスト.京都工芸繊維大学,2024年
\end{thebibliography}

\end{document}