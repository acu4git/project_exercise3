\documentclass{jlreq}

\usepackage{amsmath}
\usepackage{array}
\usepackage{bm}
\usepackage{caption}
\usepackage{fancyhdr}
\usepackage{float}
\usepackage{graphicx}
\usepackage{listings}
\usepackage{multirow}
\usepackage{physics}
\usepackage{siunitx}
\usepackage{xcolor}

\lstset{
    language=Verilog, % 使用するプログラム言語を指定
    basicstyle=\ttfamily\footnotesize, % フォントの指定
    tabsize=4, % インデント幅
    numbers=left, % 行番号を表示(必要な場合)
    numberstyle=\tiny, % 行番号のスタイル
    frame=single, % ソースコードを枠で囲む(必要な場合)
    breaklines=true, % 長い行を自動的に折り返す
    captionpos=t, % キャプションの位置を上にする
    showstringspaces=false, % 文字列内のスペースを表示しない
    keywordstyle=\color{blue}, % キーワードの色
    commentstyle=\color{green}, % コメントの色
    stringstyle=\color{red}, % 文字列の色
}
\renewcommand{\lstlistingname}{ソースコード}

\numberwithin{equation}{section}

\pagestyle{fancy}
\fancyhf{}
\fancyhead[R]{\thepage}

\begin{document}

\tableofcontents
\clearpage

\section{実験の目的}

\section{実験器具}
実験に用いた環境を以下に示す.
\begin{description}
  \item[PC] Inspiron 15 3535
  \item[OS] Windows11
  \item[CPU] AMD Ryzen 5 7530U with Radeon Graphics 2.00 GHz
  \item[統合開発環境] Quartus Prime Lite Edition Version 20.1.1
  \item[FPGAボード] Terasic DE1SoC
\end{description}

\section{理論}

\section{演習の解答}

\section{実習}

\section{考察}

\begin{thebibliography}{9}
  \bibitem{exp_text} 布目 淳.プロジェクト実習Ⅲ 論理設計 実験テキスト.京都工芸繊維大学,2024年
  \bibitem{user_manual} Terasic Technologics Inc.:  ``DE1-SoC User Manual V2.0.4'' Chapter3. Using the DE1-SoC Board, 2019
\end{thebibliography}

\end{document}